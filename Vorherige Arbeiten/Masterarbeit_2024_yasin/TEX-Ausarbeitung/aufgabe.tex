\thispagestyle{empty}
\chapter*{Aufgabenstellung}
\thispagestyle{empty}

Verfahren zum Model-Checking analoger Schaltungen basieren auf einer vollst�ndigen Diskretisierung des Zustandsraums. Diese Diskretisierung �berf�hrt den kontinuierlichen Zustandsraum in eine Graph-Datenstruktur, auf der temporallogische Model Checking - Algorithmen angewendet werden k�nnen. Diese vollst�ndige Diskretisierung des Zustandsraums vor Anwendung der Verifikationsalgorithmen ist sehr rechenzeitintensiv und abh�ngig von der auszuwertenden Spezifikation wird nur ein kleiner Teil des erzeugten Modells untersucht. 

Dementsprechend ist das Ziel dieser Arbeit, eine Methodik zu entwickeln, die abh�ngig von der zu klassifizierenden Spezifikation eine Verifikation nur auf den relevanten Teilen des Zustandsraums durchf�hrt. Dies soll auf Basis einer Abstraktionsverfeinerung durch schrittweise Exploration des Zustandsraums erfolgen, womit eine on-demand Diskretisierung der relevanten Zustandsraumgebiete durch transiente Simulationstrajektorien erreicht wird.

Als Arbeitsumgebung wird eine Schnittstelle zu einem Simulationswerkzeug f�r analoge
Schaltungen f�r die Abtastung des Zustandsraums sowie eine auf GNU Octave basierende
Entwicklungsumgebung zur Verf�gung gestellt.

