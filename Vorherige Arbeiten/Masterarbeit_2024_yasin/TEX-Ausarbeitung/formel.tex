%\thispagestyle{empty}

%\textbf{\Large{Formelzeichen}}
%\\[1.0cm]

% \addcontentsline{toc}{section}{Abk�rzungsverzeichnis}

\chapter*{Formelverzeichnis}
%\section*{Formeln}

%\begin{tabular}{p{4cm}  l}
\begin{tabular}{p{0.25\textwidth}  p{0.7\textwidth}}
\textbf{$\mathbb N$} \dotfill{} & Menge der nat�rlichen Zahlen\\ 
\textbf{$\mathbb R$} \dotfill{} & Menge der reellen Zahlen\\ 
\textbf{$\mathbb Q$} \dotfill{} & Menge der rationalen Zahlen\\ 
\\
\textbf{$\vec x$} \dotfill{} & Skalar oder Vektor\\ 
\textbf{$\vec x_i$} \dotfill{} & $i$-te Komponente des Vektors $x$\\ 
\textbf{$t$} \dotfill{} & Zeit\\ 
\textbf{$\left[UG,OG\right]$} \dotfill{} & Intervall mit Untergrenze $UG$ und Obergrenze $OG$\\ 
\textbf{$\tau$} \dotfill{} & �berschneidungsgrad zweier Bounding Boxen (in \%)\\ 
\textbf{$dim$} \dotfill{} & Dimension des Zustandsraums \\ 
\textbf{$sw_i$} \dotfill{} & Schrittweite der Transition${}_i$ \\ 
\textbf{$sw_j ^{OG}$} \dotfill{} & Schrittweitenobergrenze der Punktmengentransition${}_j$ \newline (durchschnittliche Schrittweite der Transitionen der Punktmenge) \\ 
\\
\textbf{$f(x)$} \dotfill{} & Skalar- oder vektorwertige Funktion in Abh�ngigkeit von $x$\\ 
\textbf{$\dot x(t) = \frac{dx}{dt} $} \dotfill{} & Erste Ableitung des Vektors $x$\\ 
\textbf{$f \bigl( \dot x(t), x(t), t \bigr) = 0$} \dotfill{} & Algebro-Differentialgleichungssystem \\ 
\\
%\textbf{A} \dotfill{} & Matrix\\ 
\textbf{$\mathcal A$} \dotfill{} & Menge\\ 
\textbf{$A$} \dotfill{} & Operator einer temporalen Logik\\ 
\textbf{$\Phi$} \dotfill{} & Ausdruck in temporaler Logik\\ 
\\
\textbf{$KS$} \dotfill{} & Zustandsgraph in Kripke-Struktur\\ 
\\
%$f_s$ \dotfill{} &Abtastfrequenz\\
%$\triangle f$ \dotfill{} &Spektrallinienabstand\\
%$t_{\lambda}$ \dotfill{} &Zeitdauer eines Zeitfensters\\
%$\alpha$ \dotfill{} &konstanter Gl{\"a}ttungsfaktor\\
%$\alpha (\lambda,k)$ \dotfill{} &zeit- und frequenzabh{\"a}ngiger Gl{\"a}ttungsfaktor\\
%$\alpha_c(\lambda)$ \dotfill{} &Korrektur des Gl{\"a}ttungsfaktors\\
%$\alpha_{max}$ \dotfill{} &maximaler Gl{\"a}ttungsfaktor\\
%$\alpha_{min}$ \dotfill{} &minimaler Gl{\"a}ttungsfaktor\\
%$\alpha_{opt}(\lambda,k)$ \dotfill{} &optimaler Gl{\"a}ttungsfaktor\\

%$\beta(\lambda,k)$ \dotfill{} &Gl{\"a}ttungsfaktor f{\"u}r die
%                               Momente erster und zweiter Ordnung\\
%$\epsilon$ \dotfill{} &\emph{subfloor} Konstante\\
%$\gamma$ \dotfill{} &\emph{oversubstraction factor}\\
%$\overline{\gamma}$ \dotfill{} &\emph{a posteriori SNR}\\
%$\Gamma(\cdot)$ \dotfill{} &Gammafunktion\\
%$k$ \dotfill{} &Frequenzindex\\
%$\lambda$ \dotfill{} &Zeitindex\\
%$\Omega_k$ \dotfill{} &Mittenfrequenz\\
%$\sigma^2_n$ \dotfill{} &Varianz des St{\"o}rger{\"a}usches n\\
%$\sigma^2_s$ \dotfill{} &Varianz des Sprachsignals s\\
%$\vartheta$ \dotfill{} &Vergleichsoperationen pro
%                        Zeitschritt und Frequenz\\
%$\varrho$ \dotfill{} &Steigung des Rauschleistung\\
%$\delta$ \dotfill{} &maximale Zeitverz{\"o}gerung\\
%$\chi$ \dotfill{} & Durchschnitt aller gemessenen Peakbreiten\\
%                  & f{\"u}r alle Frequenzen\\
%$\varphi_{xs}$ \dotfill{} & Kreuzkorrelationen der Signale x und s\\
%$\phi_{xx}$, $\phi_{ss}$, $\phi_{nn}$ \dotfill{}
%       &Autokorrelationen der Signale x, s, n\\
%$\Theta$ \dotfill{}
%                &Leistungsdichtespektrum\\[0.5cm]
\end{tabular}

