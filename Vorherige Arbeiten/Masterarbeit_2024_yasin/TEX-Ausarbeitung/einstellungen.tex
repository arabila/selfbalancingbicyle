%
% Einbindung von n�tigen Pakten
%
\usepackage{makeidx}         % allows index generation
%\usepackage{multicol}        % used for the two-column index
\usepackage[bottom]{footmisc}% places footnotes at page bottom

\usepackage{subfigure}
%\usepackage[bf,hang]{caption2}
\usepackage{algorithm, algorithmic} % F�r Allorithmen
\usepackage{listings}

\usepackage{url}
\newcommand{\urlBiBTeX}[1]{\url{#1}}
%\usepackage{scrpage2}						% Leider keine Ahnung, warum ich das brauche, scheint aber nicht unwichtig zu sein. Kollidiert nur leider mit fancyhdr, wegen chead & cfoot. Daher habe ich es im rahmen der Masterarbeit erstmal auskommentiert
\usepackage[dvips]{color} %pdftex
\usepackage{array}
\usepackage{fancyvrb}

\usepackage{fancyhdr}
\usepackage{german,graphicx,graphics}
%\usepackage{german,fancyhdr,graphicx,graphics}
\usepackage{amsmath,theorem,amsfonts,longtable}
%\usepackage{amsthm}
%\usepackage{enumitem}

% Beschriftung von Bildern, Tabellen, etc
\usepackage{caption3}

%\usepackage{times}
\usepackage{courier}
\usepackage{hyperref}
\usepackage{hhline}
\usepackage[latin1]{inputenc}       %% deutsche Umlaute wie normale Buchstaben verwenden\hyphenation{Sys-tem Sys-tem-mo-dell}


%
% Grundlegende Einstellungen f�r das gesamte Dokument
%
\renewcommand{\baselinestretch}{1.15}\normalsize %Zeilenabstand

\renewcommand{\captionfont}{\small\itshape}
\renewcommand{\captionlabelfont}{\small\normalfont\bfseries}
\renewcommand{\textfraction}{0.05}
\renewcommand{\topfraction}{0.95}
\renewcommand{\bottomfraction}{0.95}
\renewcommand{\floatpagefraction}{0.35}
\setcounter{totalnumber}{5}

\setlength\parskip{\medskipamount}
\setlength\parindent{0pt}

\setlength{\marginparwidth}{1.2cm}  %rechter Rand
\setlength{\textwidth}{15.0cm}    %Textbreite

\setlength{\topmargin}{-1.0cm}    %oberer Rand  -1.0
\setlength{\textheight}{22cm}   %Texth�he
\setlength{\oddsidemargin}{1.2cm} %Linker Rand

% Kleine Ver�nderung von Dictum, damit das so ausschaut, wie ich das haben m�chte
\renewcommand*{\dictumwidth}{.5\textwidth}
\renewcommand*{\raggeddictumtext}{\centering}
%\renewcommand*{\dictumtext}{\setkomafont{}{}}

% Wenn mich nicht alles t�uscht, dann ist der Teil hier f�r die sch�nen Kopfzeilen zust�ndig
\pagestyle{fancy}
\renewcommand{\sectionmark}[1]{\markright{\thesection\ #1}}			%Damit in der Kopfzeile der Section-Name nicht in Uppercase abgebildet wird
\lhead[]{\rightmark}
\rhead[]{} %\thepage
\cfoot{\thepage}
\sloppy
\setlongtables

{\theoremstyle{break}
\newtheorem {beispiel}{Beispiel}[section]
\theoremheaderfont{\scshape\bf}}

\newtheorem {definition}{Definition}[section]
\theorembodyfont{\itshape}
\theoremheaderfont{\scshape\bf}

{\theoremstyle{break}
\newtheorem {satz}{Satz}[section]
\theoremheaderfont{\scshape\bf}}

\setcounter{secnumdepth}{5} 
\setcounter{tocdepth}{3}

%
% Festlegen des Aussehens von lstlistings (Algorithmendarstellungen)
%
\definecolor{hellgrau}{gray}{0.6}				% Definiere Farbe "hellgrau"
\lstloadlanguages{Pascal}
\lstset{%
  numbers=left,										% Position der Nummerierung 
  numberstyle=\normalsize, 
  tabsize=3,
  numbersep=-10pt,               	% Verschiebung der Nummerierung, sodass sie noch im Kasten erscheint
	language=Pascal,								% Sprache ausw�hlen
%	keywordstyle=\bfseries,					% Keywords fett
%	captionpos= bottom,							% Titelposition
	frame=shadowbox,								% Bordertype
	mathescape=true,								% Mathematische Formeln anzeigen
	rulesepcolor=\color{hellgrau},  % Verwende Farbe "hellgrau" als Randfarbe
	showspaces=false,
  showtabs=false,
  columns=fixed,
  breaklines=true
}
