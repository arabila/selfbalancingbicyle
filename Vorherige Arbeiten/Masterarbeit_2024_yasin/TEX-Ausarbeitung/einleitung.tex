\chapter{Einleitung \label{ch_einleitung}}

Seit fast einem halben Jahrhundert existiert das von Gordon Moore formulierte Gesetz, dass sich die Transistoranzahl auf einem Chip innerhalb von zwei Jahren verdoppelt. 
Dieser rapide wachsenden Schaltungskomplexit�t versuchen seit jeher Industrie und Forschung mit verbesserten Entwurfsprozessen zu begegnen. Gerade der in der Industrie vorherrschende Innovationsdruck f�r neue Produkte und die immer k�rzer werdende Lebensdauer eben dieser, fordern einen Schulterschluss. Doch in der Realit�t existiert ein sogenannter, stetig wachsender \glq Design Gap\grq{} zwischen der Schaltungskomplexit�t und dem Entwurfsprozess. 
%sieht anders aus und wird im sogenannten  dargestellt. %Der \glq Design Gap\grq{} zeigt den Unterschied der theoretisch fertigbaren und der praktisch entwerfbaren Schaltungen. ODER
Abstrakt formuliert, zeigt der \glq Design Gap\grq{} den Unterschied der theoretisch fertigbaren und der praktisch entwerfbaren Schaltungen.

...


\section*{Aufbau dieser Arbeit} 

Diese Arbeit betrachtet ... 

