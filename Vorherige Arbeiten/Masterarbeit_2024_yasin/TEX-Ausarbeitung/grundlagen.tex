\chapter{Grundlagen \label{ch_grund}}

Dieses Kapitel beschreibt die Grundlagen, ...


\section{Von der Spezifikation zur Verifikation \label{sec:spezi}}

Bei der Herstellung von analogen, digitalen oder mixed-signal Schaltungen wird von der Idee bis zur Fertigstellung ein definierter Prozess durchlaufen. Dieser beinhaltet folgende Schritte:
\begin{enumerate}
	\item die Spezifikation \label{proz_spez}
	\item den Entwurf
	\item die Fertigung
	\item den Test
\end{enumerate}


...

Grammatik \eqref{eq:CTL-AT} repr�sentiert die Erweiterung der CTL - Syntax um die Neuerungen von CTL-A und CTL-AT.
\begin{align}
	\label{eq:CTL-AT}
	\Phi  \ :=  \ a  \ & | \ z \: \ast \: v \ | \ \Phi \: \circ \: \Psi \ | \ \neg \: \Phi \   \\
	& | \ \triangleright \: \diamond \: \Phi \ | \ \triangleright \: \Phi \: U \: \Psi \ | \ \triangleright \: \diamond^{-1} \: \Phi \ | \ \triangleright \: \Phi \: U^{-1} \: \Psi	 \notag \\
	& | \ \triangleright \: \diamond \: \Box \: \Phi \ | \ \triangleright \: \Phi \: U \: \Box \: \Psi \ | \ \triangleright \: \diamond^{-1} \: \Box \: \Phi \ | \ \triangleright \: \Phi \: U^{-1} \: \Box \: \Psi \notag
\end{align}
Zieht man \cite{HartongKlaHed04} f�r die Beschreibung der CTL-AT - Syntax die Grundlage der CTL-Syntax auf Tabelle \ref{tab:ctlat} , die Erweiterung aus Tabelle \ref{tab:ctlat} auf Seite \pageref{tab:ctlat}und die neu formulierten Zeitbedingungen hinzu, erh�lt man Tabelle \ref{tab:ctlat}. Die neuen Variablen und die analogen Operatoren reihen sich direkt nach der bool'schen Zustandsvariablen ein. Abgeschlossen wird die Syntaxbeschreibung nun von dem Vergangenheitsoperator und den Zeitbedingungen.
Diese werden an die temporalen Operatoren, mit Ausnahme des $X$-Operators, als Zeitintervall angeh�ngt, sodass eine minimale und maximale zeitliche Pfadl�nge definiert ist. Eine Operatorenschreibweise $\triangleright \ X \ \Box \: \Phi $ bzw. $ \triangleright \ X^{-1} \ \Box \: \Phi $ ist nicht zul�ssig.
\begin{table}[htb]
\centering
\caption[Beschreibung der CTL-AT - Syntax]{Beschreibung der CTL-AT - Syntax (Quelle: \cite[S. 10]{platte})}
\begin{tabular}{|p{0.02\textwidth}|p{0.37\textwidth}|p{0.4\textwidth}|}
\hline
$a$ & Bool'sche Zustandsvariable &  \\ \hline
$z$ & Kontinuierliche Zustandsvariable &  \\ \hline
$v$ & Reeller Zahlenwert ($ \mathbb{R} $) &  \\ \hline
\multicolumn{ 1}{|l|}{$\ast$} & \multicolumn{ 1}{l|}{Analoge Operatoren} & $\ > \ = \ $ gr��er \\ \cline{ 3- 3}
\multicolumn{ 1}{|l|}{} & \multicolumn{ 1}{l|}{} & $\ < \ = \ $ kleiner \\ \hline
\multicolumn{ 1}{|l|}{$\circ$} & \multicolumn{ 1}{l|}{Bool'sche Operatoren} & $\ \vee \ = \ $ oder \\ \cline{ 3- 3}
\multicolumn{ 1}{|l|}{} & \multicolumn{ 1}{l|}{} & $\ \wedge \ = \ $ und \\ \cline{ 1- 1}\cline{ 3- 3}
$\neg$ & \multicolumn{ 1}{l|}{} & $\ \neg \ = \ $ nicht \\ \hline
\multicolumn{ 1}{|l|}{$\triangleright$} & \multicolumn{ 1}{l|}{Pfadquantoren} & $\ A \ = \ $ auf allen Pfaden \\ \cline{ 3- 3}
\multicolumn{ 1}{|l|}{} & \multicolumn{ 1}{l|}{} &  $\ E \ = \ $ auf mindestens einem Pfad \\ \hline
\multicolumn{ 1}{|l|}{$\diamond$} & \multicolumn{ 1}{l|}{Temporale Operatoren} & $\ X \: = \ $ n�chster Zeitschritt (neXt) \\ \cline{ 3- 3}
\multicolumn{ 1}{|l|}{} & \multicolumn{ 1}{l|}{} & $\ F \ = \ $ irgendwann (Finally) \\ \cline{ 3- 3}
\multicolumn{ 1}{|l|}{} & \multicolumn{ 1}{l|}{} & $\ G \ = \ $ immer (Globally) \\ \cline{ 1- 1}\cline{ 3- 3}
$U$ & \multicolumn{ 1}{l|}{} & $\ U \ = \ $ bis (Until) \\ \hline
\multicolumn{1}{|r|}{${}^{-1}$} & Vergangenheit & $ $ Umkehrung der Zeit \\ \hline
$\Box$ & Zeitintervall & $\ \left[ t_{low}, t_{high} \right] $ mit: \newline $ \ t_{low} \in {\mathbb R}^{+} _{0}, \ t_{high} \in {\mathbb R}^{+} \ $ \newline $  t_{low}\ \leq \ t_{high}$ \\ \hline
\end{tabular}
\label{tab:ctlat}
\end{table}


...
